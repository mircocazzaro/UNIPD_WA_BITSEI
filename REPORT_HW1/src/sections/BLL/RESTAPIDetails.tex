\subsection{REST API Details}
At the end of every URI used in the Rest calls below, it will be automatically added the following path: 
\begin{lstlisting}
    company/{id}
\end{lstlisting}
which specifies the id of the company currently logged into the system. \\
Regarding the authentication aspect, when the user logs into the system a session token containing the owner id and the owner email is setted. \\
If this token is not setted, every call will be blocked until the user authenticates himself. \\
This token is particularly useful because it allows to check at every call to a Rest Resource that the user is authorized for the operation he's trying to do. \\
So, the final flow will be:
\begin{itemize}
    \item The user logs into the system using his email and password.
    \item After authentication, the \textit{JWT} token containing the owner identifier number and the owner email address is setted.
    \item After the log in page, the user company is setted by default. If an user has more companies, he can select which one to use. \\
            The company identifier is setted as a session attribute.
    \item From now on, all the pages will show only the data associated to this company.
\end{itemize}

%List here a few resources retrievable via REST API


% REST RESOURCE #1
\subsubsection*{Get the data of the user}

% the description of the resource
The following endpoint allows to get the data of the user currently logged into the system.

\begin{itemize}

    \item URL: 
    \begin{itemize}
        \item \texttt{/rest/user}
    \end{itemize}
    
    \item Method: 
    \begin{itemize}
        \item \texttt{GET}
    \end{itemize}
    
    \item URL Parameters: 
    \begin{itemize}
        \item \texttt{No URL parameters required.}
    \end{itemize}
    
    \item Data Parameters: 
    \begin{itemize}
        \item \texttt{ownerId = \{int\}} \\
        The identifier of the owner contained in the authentication token.
    \end{itemize}
    
    \item Success Response: 
    \begin{itemize}
        \item Code: 200
        \item Content: 
        \begin{lstlisting}
[
{"user":{"firstname":"Burr","lastname":"Mycah","username":"mmcclosh0","email":"mchaudret0@dailymail.co.uk","telegram_chat_id":null}},
]
        \end{lstlisting}    
    \end{itemize}
    
    \item Error Response:
    \begin{table}[!h]
    \centering 
    \begin{tabular}{|c|c|}
    \hline
    \multicolumn{1}{|c|}{\textbf{Code}} & \multicolumn{1}{c|}{\textbf{When}} \\ \hline
    Error code 1 & When of Error Code 1 \\\hline
    Error code 2 & When of Error Code 2 \\\hline
    \end{tabular} 
    \end{table} 

\end{itemize}


% REST RESOURCE #2
\subsubsection*{Reset the password of a user registered in the system}

% the description of the resource
The following endpoint allows to send a link to an user via email to reset its password.

\begin{itemize}

    \item URL: 
    \begin{itemize}
        \item \texttt{/rest/user/reset-password}
    \end{itemize}
    
    \item Method: 
    \begin{itemize}
        \item \texttt{POST}
    \end{itemize}
    
    \item URL Parameters: 
    \begin{itemize}
        \item \texttt{No URL parameters required.} 
    \end{itemize}
    
    \item Data Parameters: 
    \begin{itemize}
        \item \texttt{ownerMail = \{String\}} \\
        The mail inserted by the user in which to receive the link to reset the password.    
    \end{itemize}
    
    \item Success Response: 
    \begin{itemize}
        \item Code: 200
        \item Content:
        \begin{lstlisting}
{
    message: “Reset password token send to your email, check your inbox”
}
        \end{lstlisting}
    \end{itemize}
    
    \item Error Response:
    \begin{table}[!h]
    \centering 
    \begin{tabular}{|c|c|}
    \hline
    \multicolumn{1}{|c|}{\textbf{Code}} & \multicolumn{1}{c|}{\textbf{When}} \\ \hline
    Error code 1 & When of Error Code 1 \\\hline
    Error code 2 & When of Error Code 2 \\\hline
    \end{tabular} 
    \end{table} 
    
\end{itemize}


% REST RESOURCE #3
\subsubsection*{Change the password of a user registered in the system}

% the description of the resource
The following endpoint allows to change an user's password from inside the system.

\begin{itemize}
    
    \item URL: 
    \begin{itemize}
        \item \texttt{/rest/user/change-password}
    \end{itemize}
    
    \item Method: 
    \begin{itemize}
        \item \texttt{POST}
    \end{itemize}
    
    \item URL Parameters: 
    \begin{itemize}
        \item \texttt{No URL parameters required.} 
    \end{itemize}
    
    \item Data Parameters: 
    \begin{itemize}
        \item \texttt{ownerId = \{int\}, ownerMail = \{String\}} \\
        The owner identifier and the owner mail contained in the authentication token.    
        \item \texttt{newPassword = \{String\}} \\
        The new password to set for the current user.
        \item \texttt{resetToken = \{String\}} \\
        The reset token received via mail.
        \item 
    \end{itemize}
    
    \item Success Response: 
    \begin{itemize}
        \item Code: 200
        \item Content:
        \begin{lstlisting}
{
    message: “Successfully done, now you can login with your new password”
}
        \end{lstlisting}
    \end{itemize}
    
    \item Error Response:
    \begin{table}[!h]
    \centering 
    \begin{tabular}{|c|c|}
    \hline
    \multicolumn{1}{|c|}{\textbf{Code}} & \multicolumn{1}{c|}{\textbf{When}} \\ \hline
    Error code 1 & When of Error Code 1 \\\hline
    Error code 2 & When of Error Code 2 \\\hline
    \end{tabular} 
    \end{table} 
    
\end{itemize}


% REST RESOURCE #4
\subsubsection*{Authenticate an user}

% the description of the resource
The following endpoint allows to authenticate an user into the system.

\begin{itemize}
    
    \item URL: 
    \begin{itemize}
        \item \texttt{/rest/login}
    \end{itemize}
    
    \item Method: 
    \begin{itemize}
        \item \texttt{POST}
    \end{itemize}
    
    \item URL Parameters: 
    \begin{itemize}
        \item \texttt{No URL parameters required.} 
    \end{itemize}
    
    \item Data Parameters: 
    \begin{itemize}
        \item \texttt{The JSON representation of the user email and password.}
    \end{itemize}
    
    \item Success Response: 
    \begin{itemize}
        \item Code: 200
        \item Content:
        \begin{lstlisting}
            TODO
        \end{lstlisting}    
    \end{itemize}
    
    \item Error Response:
    \begin{table}[!h]
    \centering 
    \begin{tabular}{|c|c|}
    \hline
    \multicolumn{1}{|c|}{\textbf{Code}} & \multicolumn{1}{c|}{\textbf{When}} \\ \hline
    Error code 1 & When of Error Code 1 \\\hline
    Error code 2 & When of Error Code 2 \\\hline
    \end{tabular} 
    \end{table} 
    
\end{itemize}


% REST RESOURCE #5
\subsubsection*{List the companies associated to an user}

% the description of the resource
The following endpoint allows to list all the companies associated to an user.

\begin{itemize}
    
    \item URL: 
    \begin{itemize}
        \item \texttt{/rest/company}
    \end{itemize}
    
    \item Method: 
    \begin{itemize}
        \item \texttt{GET}
    \end{itemize}
    
    \item URL Parameters: 
    \begin{itemize}
        \item \texttt{No URL parameters required.}
    \end{itemize}
    
    \item Data Parameters: 
    \begin{itemize}
        \item \texttt{ownerId = \{int\}} \\
        The identifier of the owner contained in the authentication token.
    \end{itemize}
    
    \item Success Response: 
    \begin{itemize}
        \item Code: 200
        \item Content:
        \begin{lstlisting}
            TODO
        \end{lstlisting}    
    \end{itemize}
    
    \item Error Response:
    \begin{table}[!h]
    \centering 
    \begin{tabular}{|c|c|}
    \hline
    \multicolumn{1}{|c|}{\textbf{Code}} & \multicolumn{1}{c|}{\textbf{When}} \\ \hline
    Error code 1 & When of Error Code 1 \\\hline
    Error code 2 & When of Error Code 2 \\\hline
    \end{tabular} 
    \end{table} 
    
\end{itemize}


% REST RESOURCE #6
\subsubsection*{Create a new company associated to an user}

% the description of the resource
The following endpoint allows to create a new company and associate it to an user already registered into the system.

\begin{itemize}
    
    \item URL: 
    \begin{itemize}
        \item \texttt{/rest/company}
    \end{itemize}
    
    \item Method: 
    \begin{itemize}
        \item \texttt{POST}
    \end{itemize}
    
    \item URL Parameters: 
    \begin{itemize}
        \item \texttt{No URL parameters required.}
    \end{itemize}
    
    \item Data Parameters: 
    \begin{itemize}
        \item \texttt{ownerId = \{int\}} \\
        The identifier of the owner contained in the authentication token.    
    \end{itemize}

    \item Success Response: 
    \begin{itemize}
        \item Code: 200
        \item Content:
        \begin{lstlisting}
            TODO
        \end{lstlisting} 
    \end{itemize}
    
    \item Error Response:
    \begin{table}[!h]
    \centering 
    \begin{tabular}{|c|c|}
    \hline
    \multicolumn{1}{|c|}{\textbf{Code}} & \multicolumn{1}{c|}{\textbf{When}} \\ \hline
    Error code 1 & When of Error Code 1 \\\hline
    Error code 2 & When of Error Code 2 \\\hline
    \end{tabular} 
    \end{table} 
    
\end{itemize}


% REST RESOURCE #7
\subsubsection*{Get the image associated to a company}

% the description of the resource
The following endpoint allows to get the company image of the company specified in the URL.

\begin{itemize}
    
    \item URL: 
    \begin{itemize}
        \item \texttt{/rest/company/image/\{id\}}
    \end{itemize}
    
    \item Method: 
    \begin{itemize}
        \item \texttt{POST}
    \end{itemize}
    
    \item URL Parameters: 
    \begin{itemize}
        \item \texttt{company = \{int\}} \\
        The identifier of the company for which to retrieve the image.
    \end{itemize}
    \item Data Parameters: 
    \begin{itemize}
        \item \texttt{ownerId = \{int\}} \\
        The identifier of the owner contained in the authentication token.
    \end{itemize}
    
    \item Success Response: 
    \begin{itemize}
        \item Code: 200
        \item Content:
        \begin{lstlisting}
            TODO
        \end{lstlisting}    
    \end{itemize}
    
    \item Error Response:
    \begin{table}[!h]
    \centering 
    \begin{tabular}{|c|c|}
    \hline
    \multicolumn{1}{|c|}{\textbf{Code}} & \multicolumn{1}{c|}{\textbf{When}} \\ \hline
    Error code 1 & When of Error Code 1 \\\hline
    Error code 2 & When of Error Code 2 \\\hline
    \end{tabular} 
    \end{table} 
    
\end{itemize}


% REST RESOURCE #8
\subsubsection*{Create a new customer associated to a company}

% the description of the resource
The following endpoint allows to create a new customer and associate it to the current company logged in the system.

\begin{itemize}
    
    \item URL: 
    \begin{itemize}
        \item \texttt{/rest/customer/}
    \end{itemize}
    
    \item Method: 
    \begin{itemize}
        \item \texttt{POST}
    \end{itemize}
    
    \item URL Parameters: 
    \begin{itemize}
        \item \texttt{companyId = \{int\}} \\
        The identifier of the company to which associate the new customer.
    \end{itemize}
    
    \item Data Parameters: 
    \begin{itemize}
        \item \texttt{The JSON representation of the company to create} 
        \item \texttt{ownerId = \{int\}} \\
        The identifier of the owner contained in the authentication token.
    \end{itemize}
    
    \item Success Response: 
    \begin{itemize}
        \item Code: 200
        \item Content:
        \begin{lstlisting}
            TODO
        \end{lstlisting}    
    \end{itemize}
    
    \item Error Response:
    \begin{table}[!h]
    \centering 
    \begin{tabular}{|c|c|}
    \hline
    \multicolumn{1}{|c|}{\textbf{Code}} & \multicolumn{1}{c|}{\textbf{When}} \\ \hline
    Error code 1 & When of Error Code 1 \\\hline
    Error code 2 & When of Error Code 2 \\\hline
    \end{tabular} 
    \end{table} 
    
\end{itemize}


% REST RESOURCE #9
\subsubsection*{Get a customer associated to a company}

% the description of the resource
The following endpoint allows to get a customer associated to the current company logged in the system.

\begin{itemize}
    
    \item URL: 
    \begin{itemize}
        \item \texttt{/rest/customer/\{id\}}
    \end{itemize}
    
    \item Method: 
    \begin{itemize}
        \item \texttt{GET}
    \end{itemize}
    
    \item URL Parameters: 
    \begin{itemize}
        \item \texttt{customerId = \{int\}} \\
        The identifier of the customer to retrieve.
        \item \texttt{company = \{int\}} \\
        The identifier of the company currently logged into the system.   
    \end{itemize}
    
    \item Data Parameters: 
    \begin{itemize}
        \item \texttt{ownerId = \{int\}} \\
        The identifier of the owner contained in the authentication token.
    \end{itemize}
    
    \item Success Response: 
    \begin{itemize}
        \item Code: 200
        \item Content:
        \begin{lstlisting}
{"resource-list":
[{"customer":{"customerID":"1","businessName":"AppleLike Inc.","vatNumber":"56479-182","taxCode":"22342-305","address":"Via Venezia 79","city":"Padova","province":"PD","postalCode":null,"emailAddress":"applelike@google.com","pec":"applelike@pec-mac.com","uniqueCode":"1","companyID":"1"}}
]}
        \end{lstlisting}        
    \end{itemize}
    
    \item Error Response:
    \begin{table}[!h]
    \centering 
    \begin{tabular}{|c|c|}
    \hline
    \multicolumn{1}{|c|}{\textbf{Code}} & \multicolumn{1}{c|}{\textbf{When}} \\ \hline
    Error code 1 & When of Error Code 1 \\\hline
    Error code 2 & When of Error Code 2 \\\hline
    \end{tabular} 
    \end{table} 
    
\end{itemize}


% REST RESOURCE #10
\subsubsection*{Delete a customer associated to a company}

% the description of the resource
The following endpoint allows to delete a customer associated to the current company logged in the system.

\begin{itemize}
    
    \item URL: 
    \begin{itemize}
        \item \texttt{/rest/customer/\{id\}}
    \end{itemize}

    \item Method: 
    \begin{itemize}
        \item \texttt{DELETE}
    \end{itemize}
    
    \item URL Parameters: 
    \begin{itemize}
        \item \texttt{customerId = \{int\}} \\
        The identifier of the customer to delete.
        \item \texttt{company = \{int\}} \\
        The identifier of the company currently logged into the system.
    \end{itemize}
    
    \item Data Parameters: 
    \begin{itemize}
        \item \texttt{ownerId = \{int\}} \\
        The identifier of the owner contained in the authentication token.
    \end{itemize}
    
    \item Success Response: 
    \begin{itemize}
        \item Code: 200
        \item Content:
        \begin{lstlisting}
            TODO
        \end{lstlisting}    
    \end{itemize}
    
    \item Error Response:
    \begin{table}[!h]
    \centering 
    \begin{tabular}{|c|c|}
    \hline
    \multicolumn{1}{|c|}{\textbf{Code}} & \multicolumn{1}{c|}{\textbf{When}} \\ \hline
    Error code 1 & When of Error Code 1 \\\hline
    Error code 2 & When of Error Code 2 \\\hline
    \end{tabular} 
    \end{table} 
    
\end{itemize}


% REST RESOURCE #11
\subsubsection*{Update a customer associated to a company}

% the description of the resource
The following endpoint allows to update a customer associated to the current company logged in the system.

\begin{itemize}
    
    \item URL: 
    \begin{itemize}
        \item \texttt{/rest/customer/\{id\}}
    \end{itemize}
    
    \item Method: 
    \begin{itemize}
        \item \texttt{PUT}
    \end{itemize}
    
    \item URL Parameters: 
    \begin{itemize}
        \item \texttt{customerId = \{int\}} \\
        The identifier of the customer to delete.
        \item \texttt{company = \{int\}} \\
        The identifier of the company currently logged into the system.
    \end{itemize}
    
    \item Data Parameters: 
    \begin{itemize}
        \item \texttt{ownerId = \{int\}} \\
        The identifier of the owner contained in the authentication token.
    \end{itemize}
    
    \item Success Response: 
    \begin{itemize}
        \item Code: 200
        \item Content:
        \begin{lstlisting}
            TODO
        \end{lstlisting}    
    \end{itemize}
    
    \item Error Response:
    \begin{table}[!h]
    \centering 
    \begin{tabular}{|c|c|}
    \hline
    \multicolumn{1}{|c|}{\textbf{Code}} & \multicolumn{1}{c|}{\textbf{When}} \\ \hline
    Error code 1 & When of Error Code 1 \\\hline
    Error code 2 & When of Error Code 2 \\\hline
    \end{tabular} 
    \end{table} 
    
\end{itemize}

% REST RESOURCE #12
\subsubsection*{Create a new invoice associated to a company}

% the description of the resource
The following endpoint allows to create a new invoice and associate it to the current company logged in the system.

\begin{itemize}
    
    \item URL: 
    \begin{itemize}
        \item \texttt{/rest/invoice/}
    \end{itemize}
    
    \item Method: 
    \begin{itemize}
        \item \texttt{POST}
    \end{itemize}
    
    \item URL Parameters: 
    \begin{itemize}
        \item \texttt{companyId = \{int\}} \\
        The identifier of the company to which associate the new invoice.
    \end{itemize}
    
    \item Data Parameters: 
    \begin{itemize}
        \item \texttt{The JSON representation of the invoice to create} 
        \item \texttt{ownerId = \{int\}} \\
        The identifier of the owner contained in the authentication token.
    \end{itemize}
    
    \item Success Response: 
    \begin{itemize}
        \item Code: 200
        \item Content:
        \begin{lstlisting}
{
    "invoice": {
        "invoice_id": -1,
        "customer_id": 1,
        "status": 0,
        "warning_number": 0,
        "warning_date": "2023-03-02",
        "warning_pdf_file": "pdf_insertion_3.pdf",
        "invoice_number": "450",
        "invoice_date": "2023-03-17",
        "invoice_pdf_file": "inv_pdf_insertion_3.pdf",
        "invoice_xml_file": "inv_xml_insertion_3.xml",
        "total": 250.0,
        "discount": 3.0,
        "pension_fund_refund": 5.0,
        "has_stamp": false
    }
}
        \end{lstlisting}    
    \end{itemize}
    
    \item Error Response:
    \begin{table}[!h]
    \centering 
    \begin{tabular}{|c|c|}
    \hline
    \multicolumn{1}{|c|}{\textbf{Code}} & \multicolumn{1}{c|}{\textbf{When}} \\ \hline
    500 & When the inserted customer\_id is not present in Customer table \\\hline
    \end{tabular} 
    \end{table} 
    
\end{itemize}


% REST RESOURCE #13
\subsubsection*{Get an invoice associated to a company}

% the description of the resource
The following endpoint allows to get an invoice associated to the current company logged in the system.

\begin{itemize}
    
    \item URL: 
    \begin{itemize}
        \item \texttt{/rest/invoice/\{id\}}
    \end{itemize}
    
    \item Method: 
    \begin{itemize}
        \item \texttt{GET}
    \end{itemize}
    
    \item URL Parameters: 
    \begin{itemize}
        \item \texttt{invoiceId = \{int\}} \\
        The identifier of the invoice to retrieve.
        \item \texttt{company = \{int\}} \\
        The identifier of the company currently logged into the system.   
    \end{itemize}
    
    \item Data Parameters: 
    \begin{itemize}
        \item \texttt{ownerId = \{int\}} \\
        The identifier of the owner contained in the authentication token.
    \end{itemize}
    
    \item Success Response: 
    \begin{itemize}
        \item Code: 200
        \item Content:
        \begin{lstlisting}
{
    "invoice": {
        "invoice_id": 1,
        "customer_id": 1,
        "status": 1,
        "warning_number": 1,
        "warning_date": "2023-04-02",
        "warning_pdf_file": "pdf_update.pdf",
        "invoice_number": "400",
        "invoice_date": "2023-04-17",
        "invoice_pdf_file": "inv_pdf_update.pdf",
        "invoice_xml_file": "inv_xml_update.xml",
        "total": 200.0,
        "discount": 5.0,
        "pension_fund_refund": 3.0,
        "has_stamp": false
    }
}

        \end{lstlisting}        
    \end{itemize}
    
    \item Error Response:
    \begin{table}[!h]
    \centering 
    \begin{tabular}{|c|c|}
    \hline
    \multicolumn{1}{|c|}{\textbf{Code}} & \multicolumn{1}{c|}{\textbf{When}} \\ \hline
    500 & When the inserted invoice\_id is not present in Invoice table \\\hline
    500 & When the inserted invoice\_id is not linked to the company with company\_id \\\hline
    \end{tabular} 
    \end{table} 
    
\end{itemize}

% REST RESOURCE #14
\subsubsection*{Delete an invoice associated to a company}

% the description of the resource
The following endpoint allows to delete an invoice associated to the current company logged in the system.

\begin{itemize}
    
    \item URL: 
    \begin{itemize}
        \item \texttt{/rest/invoice/\{id\}}
    \end{itemize}

    \item Method: 
    \begin{itemize}
        \item \texttt{DELETE}
    \end{itemize}
    
    \item URL Parameters: 
    \begin{itemize}
        \item \texttt{invoiceId = \{int\}} \\
        The identifier of the invoice to delete.
        \item \texttt{company = \{int\}} \\
        The identifier of the company currently logged into the system.
    \end{itemize}
    
    \item Data Parameters: 
    \begin{itemize}
        \item \texttt{ownerId = \{int\}} \\
        The identifier of the owner contained in the authentication token.
    \end{itemize}
    
    \item Success Response: 
    \begin{itemize}
        \item Code: 200
        \item Content:
        \begin{lstlisting}
{
    "invoice": {
        "invoice_id": 34,
        "customer_id": 1,
        "status": 0,
        "warning_number": 0,
        "warning_date": "2023-03-02",
        "warning_pdf_file": "pdf_insertion_3.pdf",
        "invoice_number": "450",
        "invoice_date": "2023-03-17",
        "invoice_pdf_file": "inv_pdf_insertion_3.pdf",
        "invoice_xml_file": "inv_xml_insertion_3.xml",
        "total": 250.0,
        "discount": 3.0,
        "pension_fund_refund": 5.0,
        "has_stamp": false
    }
}
        \end{lstlisting}    
    \end{itemize}
    
    \item Error Response:
    \begin{table}[!h]
    \centering 
    \begin{tabular}{|c|c|}
    \hline
    \multicolumn{1}{|c|}{\textbf{Code}} & \multicolumn{1}{c|}{\textbf{When}} \\ \hline
    500 & When the inserted invoice\_id is not present in Invoice table \\\hline
    500 & When the inserted invoice\_id is not linked to the company with company\_id \\\hline
    \end{tabular} 
    \end{table} 
    
\end{itemize}


% REST RESOURCE #15
\subsubsection*{Update an invoice associated to a company}

% the description of the resource
The following endpoint allows to update an invoice associated to the current company logged in the system.

\begin{itemize}
    
    \item URL: 
    \begin{itemize}
        \item \texttt{/rest/invoice/\{id\}}
    \end{itemize}
    
    \item Method: 
    \begin{itemize}
        \item \texttt{GET}
    \end{itemize}
    
    \item URL Parameters: 
    \begin{itemize}
        \item \texttt{invoiceId = \{int\}} \\
        The identifier of the invoice to delete.
        \item \texttt{company = \{int\}} \\
        The identifier of the company currently logged into the system.
    \end{itemize}
    
    \item Data Parameters: 
    \begin{itemize}
	\item \texttt{The JSON representation of the invoice to update}
        \item \texttt{ownerId = \{int\}} \\
        The identifier of the owner contained in the authentication token.
    \end{itemize}
    
    \item Success Response: 
    \begin{itemize}
        \item Code: 200
        \item Content:
        \begin{lstlisting}
{
    "invoice": {
        "invoice_id": 33,
        "customer_id": 1,
        "status": 0,
        "warning_number": 1,
        "warning_date": "2023-04-02",
        "warning_pdf_file": "pdf_update.pdf",
        "invoice_number": "400",
        "invoice_date": "2023-04-17",
        "invoice_pdf_file": "inv_pdf_update.pdf",
        "invoice_xml_file": "inv_xml_update.xml",
        "total": 200.0,
        "discount": 5.0,
        "pension_fund_refund": 3.0,
        "has_stamp": false
    }
}
        \end{lstlisting}    
    \end{itemize}
    
    \item Error Response:
    \begin{table}[!h]
    \centering 
    \begin{tabular}{|c|c|}
    \hline
    \multicolumn{1}{|c|}{\textbf{Code}} & \multicolumn{1}{c|}{\textbf{When}} \\ \hline
    500 & When the inserted invoice\_id is not present in Invoice table \\\hline
    500 & When the inserted invoice\_id is not linked to the company with company\_id \\\hline
    \end{tabular} 
    \end{table} 
    
\end{itemize}







% REST RESOURCE #16
\subsubsection*{Create a new invoice product associated to an invoice, a product and a company}

% the description of the resource
The following endpoint allows to create a new invoice product related to a product and an invoice and associate it to the current company logged in the system.

\begin{itemize}
    
    \item URL: 
    \begin{itemize}
        \item \texttt{/rest/invoiceproduct//\{inv\_id\}/\{prod\_id\}}
    \end{itemize}
    
    \item Method: 
    \begin{itemize}
        \item \texttt{POST}
    \end{itemize}
    
    \item URL Parameters: 
    \begin{itemize}
	\item \texttt{invoiceId = \{int\}} \\
        The identifier of the invoice related to the invoice product to create.
	\item \texttt{productId = \{int\}} \\
        The identifier of the product related to the invoice product to create.
        \item \texttt{companyId = \{int\}} \\
        The identifier of the company to which associate the new invoice product.
    \end{itemize}
    
    \item Data Parameters: 
    \begin{itemize}
        \item \texttt{The JSON representation of the invoice product to create} 
        \item \texttt{ownerId = \{int\}} \\
        The identifier of the owner contained in the authentication token.
    \end{itemize}
    
    \item Success Response: 
    \begin{itemize}
        \item Code: 200
        \item Content:
        \begin{lstlisting}
{
    "invoiceproduct": {
        "invoice_id": 33,
        "product_id": 1,
        "quantity": 400,
        "unit_price": 119.0,
        "related_price": 12.0,
        "related_price_description": "description rel price",
        "purchase_date": "2023-03-17"
    }
}
        \end{lstlisting}    
    \end{itemize}
    
    \item Error Response:
    \begin{table}[!h]
    \centering 
    \begin{tabular}{|c|c|}
    \hline
    \multicolumn{1}{|c|}{\textbf{Code}} & \multicolumn{1}{c|}{\textbf{When}} \\ \hline
    500 & When the inserted invoice\_id and product\_id combination is not present in InvoiceProduct table \\\hline
    \end{tabular} 
    \end{table} 
    
\end{itemize}


% REST RESOURCE #17
\subsubsection*{Get an invoice product associated to an invoice, a product and a company}

% the description of the resource
The following endpoint allows to get an invoice product related to a product and an invoice and associated to the current company logged in the system.

\begin{itemize}
    
    \item URL: 
    \begin{itemize}
        \item \texttt{/rest/invoiceproduct/\{inv\_id\}/\{prod\_id\}}
    \end{itemize}
    
    \item Method: 
    \begin{itemize}
        \item \texttt{GET}
    \end{itemize}
    
    \item URL Parameters: 
    \begin{itemize}
        \item \texttt{invoiceId = \{int\}} \\
        The identifier of the invoice related to the invoice product to retrieve.
	\item \texttt{productId = \{int\}} \\
        The identifier of the product related to the invoice product to retrieve.
        \item \texttt{company = \{int\}} \\
        The identifier of the company currently logged into the system.   
    \end{itemize}
    
    \item Data Parameters: 
    \begin{itemize}
        \item \texttt{ownerId = \{int\}} \\
        The identifier of the owner contained in the authentication token.
    \end{itemize}
    
    \item Success Response: 
    \begin{itemize}
        \item Code: 200
        \item Content:
        \begin{lstlisting}
{
    "invoiceproduct": {
        "invoice_id": 1,
        "product_id": 1,
        "quantity": 50,
        "unit_price": 26.9,
        "related_price": 0.0,
        "related_price_description": null,
        "purchase_date": "2021-01-05"
    }
}
        \end{lstlisting}        
    \end{itemize}
    
    \item Error Response:
    \begin{table}[!h]
    \centering 
    \begin{tabular}{|c|c|}
    \hline
    \multicolumn{1}{|c|}{\textbf{Code}} & \multicolumn{1}{c|}{\textbf{When}} \\ \hline
    500 & When the inserted invoice\_id and product\_id combination is not present in InvoiceProduct table \\\hline
    \end{tabular} 
    \end{table} 
    
\end{itemize}

% REST RESOURCE #18
\subsubsection*{Delete an invoice product associated to an invoice, a product and a company}

% the description of the resource
The following endpoint allows to delete an invoice product related to a product and an invoice and associated to the current company logged in the system.

\begin{itemize}
    
    \item URL: 
    \begin{itemize}
        \item \texttt{/rest/invoiceproduct/\{inv\_id\}/\{prod\_id\}}
    \end{itemize}

    \item Method: 
    \begin{itemize}
        \item \texttt{DELETE}
    \end{itemize}
    
    \item URL Parameters: 
    \begin{itemize}
        \item \texttt{invoiceId = \{int\}} \\
        The identifier of the invoice related to the invoice product to delete.
	\item \texttt{productId = \{int\}} \\
        The identifier of the product related to the invoice product to delete.
        \item \texttt{company = \{int\}} \\
        The identifier of the company currently logged into the system.
    \end{itemize}
    
    \item Data Parameters: 
    \begin{itemize}
        \item \texttt{ownerId = \{int\}} \\
        The identifier of the owner contained in the authentication token.
    \end{itemize}
    
    \item Success Response: 
    \begin{itemize}
        \item Code: 200
        \item Content:
        \begin{lstlisting}
{
    "invoiceproduct": {
        "invoice_id": 33,
        "product_id": 1,
        "quantity": 400,
        "unit_price": 119.0,
        "related_price": 12.0,
        "related_price_description": "description rel price",
        "purchase_date": "2023-03-17"
    }
}
        \end{lstlisting}    
    \end{itemize}
    
    \item Error Response:
    \begin{table}[!h]
    \centering 
    \begin{tabular}{|c|c|}
    \hline
    \multicolumn{1}{|c|}{\textbf{Code}} & \multicolumn{1}{c|}{\textbf{When}} \\ \hline
    500 & When the inserted invoice\_id and product\_id combination is not present in InvoiceProduct table \\\hline
    \end{tabular} 
    \end{table} 
    
\end{itemize}


% REST RESOURCE #19
\subsubsection*{Update an invoice product associated to an invoice, a product and a company}

% the description of the resource
The following endpoint allows to update an invoice product related to a product and an invoice and associated to the current company logged in the system.

\begin{itemize}
    
    \item URL: 
    \begin{itemize}
        \item \texttt{/rest/invoiceproduct/\{inv\_id\}/\{prod\_id\}}
    \end{itemize}
    
    \item Method: 
    \begin{itemize}
        \item \texttt{GET}
    \end{itemize}
    
    \item URL Parameters: 
    \begin{itemize}
        \item \texttt{invoiceId = \{int\}} \\
        The identifier of the invoice related to the invoice product to update.
	\item \texttt{productId = \{int\}} \\
        The identifier of the product related to the invoice product to update.
        \item \texttt{company = \{int\}} \\
        The identifier of the company currently logged into the system.
    \end{itemize}
    
    \item Data Parameters: 
    \begin{itemize}
	\item \texttt{The JSON representation of the invoice product to create}
        \item \texttt{ownerId = \{int\}} \\
        The identifier of the owner contained in the authentication token.
    \end{itemize}
    
    \item Success Response: 
    \begin{itemize}
        \item Code: 200
        \item Content:
        \begin{lstlisting}
            {
    "invoiceproduct": {
        "invoice_id": 33,
        "product_id": 1,
        "quantity": 415,
        "unit_price": 121.0,
        "related_price": 12.0,
        "related_price_description": "description rel price V2",
        "purchase_date": "2023-03-17"
    }
}
        \end{lstlisting}    
    \end{itemize}
    
    \item Error Response:
    \begin{table}[!h]
    \centering 
    \begin{tabular}{|c|c|}
    \hline
    \multicolumn{1}{|c|}{\textbf{Code}} & \multicolumn{1}{c|}{\textbf{When}} \\ \hline
    500 & When the inserted invoice\_id and product\_id combination is not present in InvoiceProduct table \\\hline
    \end{tabular} 
    \end{table} 
    
\end{itemize}



% REST RESOURCE #12
\subsubsection*{List the invoices associated to a company}

% the description of the resource
The following endpoint allows to list all the invoices associated to the current company logged in the system.

\begin{itemize}
    
    \item URL: 
    \begin{itemize}
        \item \texttt{/rest/list-invoice}
    \end{itemize}
    
    \item Method: 
    \begin{itemize}
        \item \texttt{GET}
    \end{itemize}
    
    \item URL Parameters: 
    \begin{itemize}
        \item \texttt{company = \{int\}} \\
        The identifier of the company currently logged into the system.
    \end{itemize}
    
    \item Data Parameters: 
    \begin{itemize}
        \item \texttt{ownerId = \{int\}} \\
        The identifier of the owner contained in the authentication token.
    \end{itemize}
    
    \item Success Response: 
    \begin{itemize}
        \item Code: 200
        \item Content:
        \begin{lstlisting}
{"resource-list":
[{"invoice":{"invoice_id":1,"customer_id":1,"status":0,"warning_number":1,"warning_date":"2022-01-05","warning_pdf_file":"warning_pdf_file1.pdf","invoice_number":"1","invoice_date":"2022-02-06","invoice_pdf_file":"invoice_pdf_file1.pdf","invoice_xml_file":"invoice_xml_file1.xml","total":168.3,"discount":15.0,"pension_fund_refund":4.1,"has_stamp":false}},
{"invoice":{"invoice_id":29,"customer_id":29,"status":0,"warning_number":29,"warning_date":"2022-08-29","warning_pdf_file":"warning_pdf_file29.pdf","invoice_number":"29","invoice_date":"2022-09-14","invoice_pdf_file":"invoice_pdf_file29.pdf","invoice_xml_file":"invoice_xml_file29.xml","total":846.8,"discount":75.4,"pension_fund_refund":1.2,"has_stamp":false}}
]}
        \end{lstlisting}    
    \end{itemize}
    
    \item Error Response:
    \begin{table}[!h]
    \centering 
    \begin{tabular}{|c|c|}
    \hline
    \multicolumn{1}{|c|}{\textbf{Code}} & \multicolumn{1}{c|}{\textbf{When}} \\ \hline
    Error code 1 & When of Error Code 1 \\\hline
    Error code 2 & When of Error Code 2 \\\hline
    \end{tabular} 
    \end{table} 
    
\end{itemize}


%RR GET CUSTOMERS REPORT
\subsubsection*{List the customers associated to a company in a PDF file}

% the description of the resource
The following endpoint allows to list all the customers associated to the current company logged in the system in a pdf file.

\begin{itemize}
    
    \item URL: 
    \begin{itemize}
        \item \texttt{/rest/customerreport}
    \end{itemize}
    
    \item Method: 
    \begin{itemize}
        \item \texttt{GET}
    \end{itemize}
    
    \item URL Parameters: 
    \begin{itemize}
        \item \texttt{company = \{int\}} \\
        The identifier of the company currently logged into the system.
    \end{itemize}
    
    \item Data Parameters: 
    \begin{itemize}
        \item \texttt{ownerId = \{int\}} \\
        The identifier of the owner contained in the authentication token.
    \end{itemize}
    
    \item Success Response: 
    \begin{itemize}
        \item Code: 200
        
        \end{lstlisting}    
    \end{itemize}
    
    \item Error Response:
    \begin{table}[!h]
    \centering 
    \begin{tabular}{|c|c|}
    \hline
    \multicolumn{1}{|c|}{\textbf{Code}} & \multicolumn{1}{c|}{\textbf{When}} \\ \hline
    Error code 1 & When of Error Code 1 \\\hline
    Error code 2 & When of Error Code 2 \\\hline
    \end{tabular} 
    \end{table} 
    
\end{itemize}



%RR GET CUSTOMERS REPORT
\subsubsection*{List the products associated to a company in a PDF file}

% the description of the resource
The following endpoint allows to list all the products associated to the current company logged in the system in a PDF file.

\begin{itemize}
    
    \item URL: 
    \begin{itemize}
        \item \texttt{/rest/customerreport}
    \end{itemize}
    
    \item Method: 
    \begin{itemize}
        \item \texttt{GET}
    \end{itemize}
    
    \item URL Parameters: 
    \begin{itemize}
        \item \texttt{company = \{int\}} \\
        The identifier of the company currently logged into the system.
    \end{itemize}
    
    \item Data Parameters: 
    \begin{itemize}
        \item \texttt{ownerId = \{int\}} \\
        The identifier of the owner contained in the authentication token.
    \end{itemize}
    
    \item Success Response: 
    \begin{itemize}
        \item Code: 200
        
        \end{lstlisting}    
    \end{itemize}
    
    \item Error Response:
    \begin{table}[!h]
    \centering 
    \begin{tabular}{|c|c|}
    \hline
    \multicolumn{1}{|c|}{\textbf{Code}} & \multicolumn{1}{c|}{\textbf{When}} \\ \hline
    Error code 1 & When of Error Code 1 \\\hline
    Error code 2 & When of Error Code 2 \\\hline
    \end{tabular} 
    \end{table} 
    
\end{itemize}




% REST RESOURCE #13
\subsubsection*{List the customers associated to a company}

% the description of the resource
The following endpoint allows to list all the customers associated to the current company logged in the system.

\begin{itemize}
    
    \item URL: 
    \begin{itemize}
        \item \texttt{/rest/list-customer}
    \end{itemize}
    
    \item Method: 
    \begin{itemize}
        \item \texttt{GET}
    \end{itemize}
    
    \item URL Parameters: 
    \begin{itemize}
        \item \texttt{company = \{int\}} \\
        The identifier of the company currently logged into the system.
    \end{itemize}
    
    \item Data Parameters: 
    \begin{itemize}
        \item \texttt{ownerId = \{int\}} \\
        The identifier of the owner contained in the authentication token.
    \end{itemize}
    
    \item Success Response: 
    \begin{itemize}
        \item Code: 200
        \item Content:
        \begin{lstlisting}
{"resource-list":
[{"customer":{"customerID":"1","businessName":"AppleLike Inc.","vatNumber":"56479-182","taxCode":"22342-305","address":"Via Venezia 79","city":"Padova","province":"PD","postalCode":null,"emailAddress":"applelike@google.com","pec":"applelike@pec-mac.com","uniqueCode":"1","companyID":"1"}},
{"customer":{"customerID":"2","businessName":"Reinger Group","vatNumber":"49631-182","taxCode":"14783-305","address":"4 Mockingbird Junction","city":"Rovigo","province":"RO","postalCode":45100,"emailAddress":"rfrankum1@google.nl","pec":"plyster1@pec-mac.com","uniqueCode":"2","companyID":"2"}}
]}
        \end{lstlisting}    
    \end{itemize}
    
    \item Error Response:
    \begin{table}[!h]
    \centering 
    \begin{tabular}{|c|c|}
    \hline
    \multicolumn{1}{|c|}{\textbf{Code}} & \multicolumn{1}{c|}{\textbf{When}} \\ \hline
    Error code 1 & When of Error Code 1 \\\hline
    Error code 2 & When of Error Code 2 \\\hline
    \end{tabular} 
    \end{table} 
    
\end{itemize}


% REST RESOURCE #14
\subsubsection*{List the products associated to a company}

% the description of the resource
The following endpoint allows to list all the products associated to the current company logged in the system.

\begin{itemize}
    
    \item URL: 
    \begin{itemize}
        \item \texttt{/rest/list-product}
    \end{itemize}
    
    \item Method: 
    \begin{itemize}
        \item \texttt{GET}
    \end{itemize}
    
    \item URL Parameters: 
    \begin{itemize}
        \item \texttt{company = \{int\}} \\
        The identifier of the company currently logged into the system.
    \end{itemize}
    
    \item Data Parameters: 
    \begin{itemize}
        \item \texttt{ownerId = \{int\}} \\
        The identifier of the owner contained in the authentication token.
    \end{itemize}

    \item Success Response: 
    \begin{itemize}
        \item Code: 200
        \item Content:
        \begin{lstlisting}
{"resource-list":
[{"product":{"product_id":1,"company_id":1,"title":"ETHANOL","default_price":29,"logo":"http://dummyimage.com/222x224.png/5fa2dd/ffffff","measurement_unit":"Kg","description":"Pre-emptive upward-trending analyzer"}},
{"product":{"product_id":2,"company_id":2,"title":"Antimoium crud, Benzoic ac, Ledum, Nux vom, Quercus, Rhododendron, Silicea","default_price":24,"logo":"http://dummyimage.com/225x208.png/5fa2dd/ffffff","measurement_unit":"Kg","description":"Ameliorated mission-critical adapter"}}
]}
        \end{lstlisting}    
    \end{itemize}

    \item Error Response:
    \begin{table}[!h]
    \centering 
    \begin{tabular}{|c|c|}
    \hline
    \multicolumn{1}{|c|}{\textbf{Code}} & \multicolumn{1}{c|}{\textbf{When}} \\ \hline
    Error code 1 & When of Error Code 1 \\\hline
    Error code 2 & When of Error Code 2 \\\hline
    \end{tabular} 
    \end{table} 
    
\end{itemize}


% REST RESOURCE #15
\subsubsection*{List the filtered invoices associated to a company}

% the description of the resource
The following endpoint allows to filter and list the invoices associated to the current company logged in the system.

\begin{itemize}
    
    \item URL: 
    \begin{itemize}
        \item \texttt{/rest/filter-invoices}
    \end{itemize}
    
    \item Method: 
    \begin{itemize}
        \item \texttt{POST}
    \end{itemize}
    
    \item URL Parameters: 
    \begin{itemize}
        \item \texttt{company = \{int\}} \\
        The identifier of the company currently logged into the system.
    \end{itemize}

    \item Data Parameters: 
    \begin{itemize}
        \item \texttt{ownerId = \{int\}} \\
        The identifier of the owner contained in the authentication token.
        \item the user can set one, many or all the filters defined below: 
        \begin{itemize}
            \item \texttt{fromTotal = \{Double\}} - the lower bound for the \textit{Total},
            \item \texttt{toTotal = \{Double\}} - the upper bound for the \textit{Total},
            \item \texttt{fromDiscount = \{Double\}} - the lower bound for the \textit{Discount}, 
            \item \texttt{toDiscount = \{Double\}} - the upper bound for the \textit{Discount},
            \item \texttt{fromPfr = \{Double\}} - the lower bound for the \textit{Pension Fund Refund}, 
            \item \texttt{toPfr = \{Double\}} - the upper bound for the \textit{Pension Fund Refund},
            \item \texttt{fromInvoiceDate = \{Date\}} - the lower bound for the \textit{Invoice Date}, 
            \item \texttt{toInvoiceDate = \{Date\}} - the upper bound for the \textit{Invoice Date},
            \item \texttt{fromWarningDate = \{Date\}} - the lower bound for the \textit{Warning Date}, 
            \item \texttt{toWarningDate = \{Date\}} - the upper bound for the \textit{Warning Date},
            \item \texttt{fromBusinessName = \{String(1)---String(2)---...---String(n)\}} - the list of the \textit{Customer Names} to filter the invoices,
            \item \texttt{fromProductTitle = \{String(1)---String(2)---...---String(n)\}} - the list of the \textit{Product Titles} to filter the invoices,
        \end{itemize}     
    \end{itemize}
    
    \item Success Response: 
    \begin{itemize}
        \item Code: 200
        \item Content:
        \begin{lstlisting}
{"resource-list":
[{"invoice":{"invoice_id":1,"customer_id":1,"status":0,"warning_number":1,"warning_date":"2022-01-05","warning_pdf_file":"warning_pdf_file1.pdf","invoice_number":"1","invoice_date":"2022-02-06","invoice_pdf_file":"invoice_pdf_file1.pdf","invoice_xml_file":"invoice_xml_file1.xml","total":168.3,"discount":15.0,"pension_fund_refund":4.1,"has_stamp":false}},
{"invoice":{"invoice_id":20,"customer_id":20,"status":0,"warning_number":20,"warning_date":"2022-09-17","warning_pdf_file":"warning_pdf_file20.pdf","invoice_number":"20","invoice_date":"2022-09-20","invoice_pdf_file":"invoice_pdf_file20.pdf","invoice_xml_file":"invoice_xml_file20.xml","total":71.2,"discount":46.8,"pension_fund_refund":3.2,"has_stamp":false}}
]}
        \end{lstlisting}    
    \end{itemize}
    
    \item Error Response:
    \begin{table}[!h]
    \centering 
    \begin{tabular}{|c|c|}
    \hline
    \multicolumn{1}{|c|}{\textbf{Code}} & \multicolumn{1}{c|}{\textbf{When}} \\ \hline
    Error code 1 & When of Error Code 1 \\\hline
    Error code 2 & When of Error Code 2 \\\hline
    \end{tabular} 
    \end{table} 
    
\end{itemize}

% REST RESOURCE #16
\subsubsection*{Plot the filtered charts associated to a company}

% the description of the resource
The following endpoint allows to filter and list the invoices associated to the current company logged in the system and plot a chart.

\begin{itemize}
    
    \item URL: 
    \begin{itemize}
        \item \texttt{/rest/charts}
    \end{itemize}
    
    \item Method: 
    \begin{itemize}
        \item \texttt{POST}
    \end{itemize}
    
    \item URL Parameters: 
    \begin{itemize}
        \item \texttt{company = \{int\}} \\
        The identifier of the company currently logged into the system.
    \end{itemize}

    \item Data Parameters: 
    \begin{itemize}
        \item \texttt{ownerId = \{int\}} \\
        The identifier of the owner contained in the authentication token.
        \item the user can set one, many or all the filters defined below: 
        \begin{itemize}
            \item \texttt{fromTotal = \{Double\}} - the lower bound for the \textit{Total},
            \item \texttt{toTotal = \{Double\}} - the upper bound for the \textit{Total},
            \item \texttt{fromDiscount = \{Double\}} - the lower bound for the \textit{Discount}, 
            \item \texttt{toDiscount = \{Double\}} - the upper bound for the \textit{Discount},
            \item \texttt{fromPfr = \{Double\}} - the lower bound for the \textit{Pension Fund Refund}, 
            \item \texttt{toPfr = \{Double\}} - the upper bound for the \textit{Pension Fund Refund},
            \item \texttt{fromInvoiceDate = \{Date\}} - the lower bound for the \textit{Invoice Date}, 
            \item \texttt{toInvoiceDate = \{Date\}} - the upper bound for the \textit{Invoice Date},
            \item \texttt{fromWarningDate = \{Date\}} - the lower bound for the \textit{Warning Date}, 
            \item \texttt{toWarningDate = \{Date\}} - the upper bound for the \textit{Warning Date},
            \item \texttt{fromBusinessName = \{String(1)---String(2)---...---String(n)\}} - the list of the \textit{Customer Names} to filter the invoices,
            \item \texttt{fromProductTitle = \{String(1)---String(2)---...---String(n)\}} - the list of the \textit{Product Titles} to filter the invoices,
            \item \texttt{chart\_type = \{int\}} - the chart type: 1 for Invoice by date, 2 for Total by date, 3 for Discount by date, 4 for Invoice by custmer, 5 for Total by customer,
            \item \texttt{chart\_period = \{int\}} - the chart period we want to group dates in the x-axis: 1 for Months, 2 for Days, 3 for Years
        \end{itemize}     
    \end{itemize}
    
    \item Success Response: 
    \begin{itemize}
        \item Code: 200
        \item Content:
        \begin{lstlisting}
{}
        \end{lstlisting}    
    \end{itemize}
    
    \item Error Response:
    \begin{table}[!h]
    \centering 
    \begin{tabular}{|c|c|}
    \hline
    \multicolumn{1}{|c|}{\textbf{Code}} & \multicolumn{1}{c|}{\textbf{When}} \\ \hline
    Error code 1 & When of Error Code 1 \\\hline
    Error code 2 & When of Error Code 2 \\\hline
    \end{tabular} 
    \end{table} 
    
\end{itemize}


\subsubsection*{bankaccount resurce}

% Bank account resurce

Bank account contains the Bank account details.

\begin{itemize}
    \item URL: \texttt{/rest/bankaccount/\{bankaccount ID\}/company/\{company ID\}}
    \item Method: \texttt{GET}
    \item URL Parameters: 
        \begin{itemize}
            \item bankaccount ID:\{int\} bankaccount ID in the database
            \item company ID:\{int\} company ID in the database
        \end{itemize}
    \item Data Parameters: None
    \item Success Response: 
    \begin{itemize}
        \item Code: 200
        \item Content: example:
        \begin{lstlisting}
        {
        "bankaccount_id": 1,
        "IBAN": "IT25674893201252722926",
        "bank_name": "ISP",
        "bankaccount_friendly_name": "Intesa San Paolo",
        "company_id": 1
        }
        \end{lstlisting}   
    \end{itemize}
        \item Error Response:
        \begin{table}[!h]
        \centering 
        \begin{tabular}{|c|c|}
        \hline
        \multicolumn{1}{|c|}{\textbf{Code}} & \multicolumn{1}{c|}{\textbf{When}} \\ \hline
        500 & If bank account not found \\\hline
        500 & If any database related erros accours\\\hline
        \end{tabular} 
        \end{table} 
    
\end{itemize}


\subsubsection*{A resource}

%TODO remember to remove this template
% the description of the resource

Lorem ipsum dolor sit amet, consectetur adipiscing elit, sed do eiusmod tempor incididunt ut labore et dolore magna aliqua. Ut enim ad minim veniam, quis nostrud exercitation ullamco laboris nisi ut aliquip ex ea commodo consequat. Duis aute irure dolor in reprehenderit in voluptate velit esse cillum dolore eu fugiat nulla pariatur. Excepteur sint occaecat cupidatat non proident, sunt in culpa qui officia deserunt mollit anim id est laborum.


\begin{itemize}
    \item URL: \texttt{the URL to retrieve it}
    \item Method: \texttt{Method to retrieve it}
    \item URL Parameters:
    \item Data Parameters: 
    \item Success Response:
    \item Error Response:
    
\end{itemize}


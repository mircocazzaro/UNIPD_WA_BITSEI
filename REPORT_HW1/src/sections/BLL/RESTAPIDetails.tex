\subsection{REST API Details}

%List here a few resources retrievable via REST API
\subsubsection*{List the users registered in the system}

% the description of the resource
The following endpoint allows to get a list of all the users registered in the system.

\begin{itemize}
    \item URL: \texttt{/rest/user}
    \item Method: \texttt{GET}
    \item URL Parameters: \texttt{No URL parameters required}
    \item Data Parameters: \texttt{No Data parameters required}
    \item Success Response: 
    \begin{itemize}
        \item Code: 200
        \item Content: 
        \begin{lstlisting}
{"resource-list":
[
{"user":{"firstname":"Burr","lastname":"Mycah","username":"mmcclosh0","email":"mchaudret0@dailymail.co.uk","telegram_chat_id":null}},
{"user":{firstname":"Daffy","lastname":"Lisette","username":"ldinningtont","email":"lgianettinit@wp.com","telegram_chat_id":null}}
]} 
        \end{lstlisting}
        
    \end{itemize}
    \item Error Response:
    \begin{table}[!h]
    \centering 
    \begin{tabular}{|c|c|}
    \hline
    \multicolumn{1}{|c|}{\textbf{Code}} & \multicolumn{1}{c|}{\textbf{When}} \\ \hline
    Error code 1 & When of Error Code 1 \\\hline
    Error code 2 & When of Error Code 2 \\\hline
    \end{tabular} 
    \end{table} 
    
\end{itemize}


\subsubsection*{Reset the password of a user registered in the system}

% the description of the resource
The following endpoint allows to send a link to an user via email to reset its password.

\begin{itemize}
    \item URL: \texttt{/rest/user/reset-password}
    \item Method: \texttt{POST}
    \item URL Parameters: \texttt{No URL parameters required}
    \item Data Parameters: \texttt{The JSON representation of the user}
    \item Success Response: 
    \begin{itemize}
        \item Code: 200
        \item Content:
        \begin{lstlisting}
            TODO
        \end{lstlisting}
        
    \end{itemize}
    \item Error Response:
    \begin{table}[!h]
    \centering 
    \begin{tabular}{|c|c|}
    \hline
    \multicolumn{1}{|c|}{\textbf{Code}} & \multicolumn{1}{c|}{\textbf{When}} \\ \hline
    Error code 1 & When of Error Code 1 \\\hline
    Error code 2 & When of Error Code 2 \\\hline
    \end{tabular} 
    \end{table} 
    
\end{itemize}


\subsubsection*{Change the password of a user registered in the system}

% the description of the resource
The following endpoint allows to change an user's password from inside the system.

\begin{itemize}
    \item URL: \texttt{/rest/user/change-password}
    \item Method: \texttt{POST}
    \item URL Parameters: \texttt{No URL parameters required}
    \item Data Parameters: \texttt{The JSON representation of the user}
    \item Success Response: 
    \begin{itemize}
        \item Code: 200
        \item Content:
        \begin{lstlisting}
            TODO
        \end{lstlisting}
        
    \end{itemize}
    \item Error Response:
    \begin{table}[!h]
    \centering 
    \begin{tabular}{|c|c|}
    \hline
    \multicolumn{1}{|c|}{\textbf{Code}} & \multicolumn{1}{c|}{\textbf{When}} \\ \hline
    Error code 1 & When of Error Code 1 \\\hline
    Error code 2 & When of Error Code 2 \\\hline
    \end{tabular} 
    \end{table} 
    
\end{itemize}


\subsubsection*{Authenticate an user}

% the description of the resource
The following endpoint allows to authenticate an user into the system.

\begin{itemize}
    \item URL: \texttt{/rest/user/change-password}
    \item Method: \texttt{POST}
    \item URL Parameters: \texttt{No URL parameters required}
    \item Data Parameters: \texttt{The JSON representation of the user}
    \item Success Response: 
    \begin{itemize}
        \item Code: 200
        \item Content:
        \begin{lstlisting}
            TODO
        \end{lstlisting}
        
    \end{itemize}
    \item Error Response:
    \begin{table}[!h]
    \centering 
    \begin{tabular}{|c|c|}
    \hline
    \multicolumn{1}{|c|}{\textbf{Code}} & \multicolumn{1}{c|}{\textbf{When}} \\ \hline
    Error code 1 & When of Error Code 1 \\\hline
    Error code 2 & When of Error Code 2 \\\hline
    \end{tabular} 
    \end{table} 
    
\end{itemize}


\subsubsection*{List the companies associated to an user}

% the description of the resource
The following endpoint allows to list all the companies associated to an user.

\begin{itemize}
    \item URL: \texttt{/rest/user/company}
    \item Method: \texttt{GET}
    \item URL Parameters: \texttt{No URL parameters required}
    \item Data Parameters: \texttt{The JSON representation of the user}
    \item Success Response: 
    \begin{itemize}
        \item Code: 200
        \item Content:
        \begin{lstlisting}
            TODO
        \end{lstlisting}
        
    \end{itemize}
    \item Error Response:
    \begin{table}[!h]
    \centering 
    \begin{tabular}{|c|c|}
    \hline
    \multicolumn{1}{|c|}{\textbf{Code}} & \multicolumn{1}{c|}{\textbf{When}} \\ \hline
    Error code 1 & When of Error Code 1 \\\hline
    Error code 2 & When of Error Code 2 \\\hline
    \end{tabular} 
    \end{table} 
    
\end{itemize}


\subsubsection*{Create a new company associated to an user}

% the description of the resource
The following endpoint allows to create a new company and associate it to an user already registered into the system.

\begin{itemize}
    \item URL: \texttt{/rest/user/company}
    \item Method: \texttt{POST}
    \item URL Parameters: \texttt{No URL parameters required}
    \item Data Parameters: \texttt{The JSON representation of the user}
    \item Success Response: 
    \begin{itemize}
        \item Code: 200
        \item Content:
        \begin{lstlisting}
            TODO
        \end{lstlisting}
        
    \end{itemize}
    \item Error Response:
    \begin{table}[!h]
    \centering 
    \begin{tabular}{|c|c|}
    \hline
    \multicolumn{1}{|c|}{\textbf{Code}} & \multicolumn{1}{c|}{\textbf{When}} \\ \hline
    Error code 1 & When of Error Code 1 \\\hline
    Error code 2 & When of Error Code 2 \\\hline
    \end{tabular} 
    \end{table} 
    
\end{itemize}

\subsubsection*{A resource}

% the description of the resource

Lorem ipsum dolor sit amet, consectetur adipiscing elit, sed do eiusmod tempor incididunt ut labore et dolore magna aliqua. Ut enim ad minim veniam, quis nostrud exercitation ullamco laboris nisi ut aliquip ex ea commodo consequat. Duis aute irure dolor in reprehenderit in voluptate velit esse cillum dolore eu fugiat nulla pariatur. Excepteur sint occaecat cupidatat non proident, sunt in culpa qui officia deserunt mollit anim id est laborum.


\begin{itemize}
    \item URL: \texttt{the URL to retrieve it}
    \item Method: \texttt{Method to retrieve it}
    \item URL Parameters:
    \item Data Parameters: 
    \item Success Response:
    \item Error Response:
    
\end{itemize}

